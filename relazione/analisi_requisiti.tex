\section*{Analisi dei requisiti}
Si vuole realizzare una base di dati per la gestione di una libreria musicale
condivisa e la relativa interfaccia web che permetta interazione tra gli
utenti.\\
Il cuore della libreria è formato da un insieme di album, suddivisi a loro
volta in brani musicali; ogni album è identificato da un codice, ed è formato
da alcuni metadati(titolo, autore, anno di pubblicazione) ed è specificato se
si tratta di un album registrato in studio o una versione live, in quest'ultimo
caso è possibile specificare la città in cui si è svolto il concerto, possiede
inoltre informazioni opzionali di carattere generale (critiche ricevute,
recensioni o breve storia sulla realizzazione dell'album).
Gli album possono avere una copertina, a cui fanno riferimento anche tutti i
brani che contengono.\\ Ogni brano musicale contenuto nell'album è
identificato da un codice, ed è formato da alcuni metadati quali titolo,
genere, durata.
Esistono due tipi di utenti che possono accedere alla libreria, ordinari e
amministratori, di entrambi interessano l'indirizzo e-mail, uno username e una
password, sono opzionali i dati anagrafici quali nome e cognome. Gli utenti
odinari possono seguire altri utenti ordinari, eccetto se stessi, inoltre ogni
utente ha la possibilità di creare una propria collezione di brani preferiti
selezionandoli dalla libreria, creare una coda di riproduzione anonima, o
creare delle playlist delle quali interessa sapere il nome, queste ultime
possono inoltre essere condivise con altri utenti \lq\lq seguiti \rq\rq.
All'interno della collezione i brani non possono ripetersi mentre nelle code di
riproduzione o nelle playlist uno stesso brano può comparire più volte.
All'atto di registrazione un utente può decidere se attivare un abbonamento
free o utilizzare un piano premium.
