\documentclass[11pt,a4paper]{article}
\usepackage[utf8]{inputenc}
\usepackage{alltt}
\usepackage{caption}
\usepackage{listings}
\usepackage{xcolor}
\usepackage{graphicx}
\usepackage{lmodern}
\usepackage[top=2in, bottom=1.5in, left=0.7in, right=0.7in]{geometry}

\DeclareCaptionFormat{listing}{\rule{\dimexpr\textwidth+17pt\relax}{0.4}\vskip1pt#1#2#3}
\captionsetup[lstlisting]{singlelinecheck=false, margin=0pt,
	font={bf,footnotesize}}

\title{Grooveclam}
\author{Andrea Giacomo Baldan 579117}

\begin{document}
\maketitle
\begin{abstract}
	A seguito degli eventi riguardanti il caso 'Napster' nei primi anni 2000,
	l'industria musicale e la distribuzione del materiale digitale ha subito
	notevoli cambiamenti e negli anni successivi prese piede il fenomeno del
	P2P (scambio tra utenti di files musicali, e non solo, mediante la rete)
	avviato da 'Napster', seguito da piattaforme e siti che offrono un servizio
	di streaming di file audio/video nel (quasi) totale rispetto dei diritti
	sugli album pubblicati. Grooveclam è una piattaforma online sulla linea del
	recente defunto Grooveshark, un sito di streaming audio appunto, e si
	propone di offrire un servizio di condivisione musicale tra utenti,
	permettendo di selezionare brani MP3 per l'ascolto, organizzarli in
	playlist che possono essere condivise tra utenti connessi tra di loro o
	semplici code di riproduzione anonime, offrendo la possibilità di generare
	e popolare la propria libreria personale di brani e di contribuire al
	popolamento della base di dati su cui poggia la piattaforma aggiungendo le
	proprie canzoni e rendendole disponibili per l'ascolto a tutti gli utenti.
\end{abstract}
\end{document}
